\documentclass[11pt,letterpaper]{article}
\usepackage[utf8]{inputenc}
\usepackage[francais]{babel} %TODO
\usepackage{amsmath}
\usepackage{amsthm}
\usepackage{amsfonts}
\usepackage{amssymb}
\usepackage{graphicx}
\usepackage{hyperref}
\usepackage{url}
\usepackage{multirow}
\usepackage{color}
\usepackage{tikz}
\usepackage{subcaption}

%\usepackage[left=2.5cm,right=2.5cm,top=3cm,bottom=3cm]{geometry}




%%%%%%%%%%%%%%%%%%%%%%%%%%%%%%%%%%%%%%%%%%%%%%%%%%%%%%%%%%%%%%%%%%%%%%%%%%%%%%%%%%%%%%%%%%%%%%
% Def - begin
%TODO
\def\name{Romain Vanden Bulcke}
\def\matricule{\texttt{1971718}}
\def\coursename{Programmation Mathématique I}
\def\courselabel{MTH6403}
\def\hwkname{Partie II - Optimisation  Non-Linéaire}
\def\session{Automne 2018}
\def\datereturn{26 Octobre  2018}
\def\prof{Charles Audet}
\def\school{Ecole Polytechnique de Montréal}
\def\logo{pics/polytechnique_genie_gauche_eng_rgb.png}

\newtheorem{theorem}{Théorème}[section]
\newtheorem{lemma}[theorem]{Lemme}
\newtheorem{prop}[theorem]{Proposition}
\newtheorem{definition}{Définition}[section]
\newtheorem{exmp}{Exemple}[section]
\newtheorem*{remark}{Remarque}
 
%TODO

% Operators and title
\def\ie{\textit{i.e. }}
\def\tabsize{4}
\DeclareMathOperator{\argmax}{arg\max}
\DeclareMathOperator{\vect}{\text{vect}}
\DeclareMathOperator{\tab}{\hspace{\tabsize mm}}
\author{\name}
\title{\courselabel - \coursename : \hwkname}
% Print title - begin
\def\printitle{
\begin{center}
{\huge \courselabel - \coursename}\\
{\LARGE \hwkname}\\
\end{center}
\hrulefill
}
% Print title - end

%%%%%%%%%%%%%%%%%%%%%%%%%%%%%%%%%%%%%%%%%%%%%%%
% Ending
\def\ending{
\noindent\hfil\rule{0.75\textwidth}{.4pt}\hfil}
%%%%%%%%%%%%%%%%%%%%%%%%%%%%%%%%%%%%%%%%%%%%%%%
% Document - begin
